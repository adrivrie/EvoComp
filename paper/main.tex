%%%%%%%%%%%%%%%%%%%%%%%%%%%%%%%%%%%%%%%%%%%%%%%%%%%%%%%%%%%%%%%%%%%%%%%%%%%%%%%%
%2345678901234567890123456789012345678901234567890123456789012345678901234567890
%        1         2         3         4         5         6         7         8

\documentclass[letterpaper, 10 pt, conference]{ieeeconf}  % Comment this line out
                                                          % if you need a4paper
%\documentclass[a4paper, 10pt, conference]{ieeeconf}      % Use this line for a4
                                                          % paper

\IEEEoverridecommandlockouts                              % This command is only
                                                          % needed if you want to
                                                          % use the \thanks command
\overrideIEEEmargins
% See the \addtolength command later in the file to balance the column lengths
% on the last page of the document



% The following packages can be found on http:\\www.ctan.org
%\usepackage{graphics} % for pdf, bitmapped graphics files
%\usepackage{epsfig} % for postscript graphics files
%\usepackage{mathptmx} % assumes new font selection scheme installed
%\usepackage{times} % assumes new font selection scheme installed
%\usepackage{amsmath} % assumes amsmath package installed
%\usepackage{amssymb}  % assumes amsmath package installed
\usepackage{caption,booktabs,array}

\title{\LARGE \bf
Island Model to Optimise the Katsuura Function
}

%\author{ \parbox{3 in}{\centering Huibert Kwakernaak*
%         \thanks{*Use the $\backslash$thanks command to put information here}\\
%         Faculty of Electrical Engineering, Mathematics and Computer Science\\
%         University of Twente\\
%         7500 AE Enschede, The Netherlands\\
%         {\tt\small h.kwakernaak@autsubmit.com}}
%         \hspace*{ 0.5 in}
%         \parbox{3 in}{ \centering Pradeep Misra**
%         \thanks{**The footnote marks may be inserted manually}\\
%        Department of Electrical Engineering \\
%         Wright State University\\
%         Dayton, OH 45435, USA\\
%         {\tt\small pmisra@cs.wright.edu}}
%}

\author{A., S., O., M.}


\begin{document}



\maketitle
\thispagestyle{empty}
\pagestyle{empty}


%%%%%%%%%%%%%%%%%%%%%%%%%%%%%%%%%%%%%%%%%%%%%%%%%%%%%%%%%%%%%%%%%%%%%%%%%%%%%%%%
\begin{abstract}

The abstract goes here.  It should be about 200 words and give the
reader a summary of the main contributions of the paper.   
Remember that readers may decide to read or not to read your
paper based on what is in the abstract.  The abstract never
contains references.  

\end{abstract}

\section{Introduction}

Very (VERY) brief introduction to EC/function maximization.

Research question(s) (island model, parameter tuning, GAVaPS, katsuura).

Optional: sign-post paragraph (section 16 will cover so-and-so)


\section{Algorithms}

Quick introduction into how Evo Algs work, including our specific methods (islands / crossover). Explanations can be slightly shallow (don't need to say migration gets picked from top 50\%), so long as this gets cleared up by our parameter settings.

Parameter settings.

Discussion about GAVaPS.

\begin{table}[h]
\caption{Parameters subject to parameter tuning.}
\label{tab:tuningparams}
\centering
\begin{tabular}{>{\quad}ll}
\toprule
\textbf{Parameter} & \textbf{Search range} \\
\midrule
Number of islands   & $[1,500]$ \\
Island size & $[10,2000]$ \\
Migration size      & $[0.1\%,20\%]$ \\
Crossover rate $p_c$ & $[0,1]$ \\
\bottomrule
\end{tabular}
\end{table}

\begin{table}[h]
\caption{Fixed symbolic and relevant numeric parameters of the algorithm.}
\label{tab:params}
\centering
\begin{tabular}{>{\quad}ll}
\toprule
\textbf{Parameter} & \textbf{Value} \\
\midrule
Representation      & Real-valued vector \\
\midrule
Recombination       & Discrete \\
\midrule
Mutation            & Self-adaptive (uncorrelated) \\
\quad General lr. $\tau$ & $\sqrt{200}^{-1}$ \\
\quad Specific lr. $\tau'$ & $\sqrt{20}^{-1}$ \\
\midrule
Parent selection    & Uniform random \\
\quad Reproduction ratio $\rho=\lambda:\mu$) & 4 \\
\midrule
Survivor selection  & ($\mu,\lambda$) with elitism \\
\quad Elite size & 1 \\
\midrule
Migration           & Circle \\
\quad Timing        & After convergence \\
\quad Convergence threshold & 15 gen. without improvement \\
\quad Immigrant selection & Uniform random from fittest half \\
\midrule
Initialisation      & Uniform random \\
\midrule
Termination         & After 1.000.000 fitness evaluations \\
\bottomrule
\end{tabular}
\end{table}

\section{Methodology}

Random search over parameter space (with short justification).

Gather interesting information about this random search, and get a good parameter setting out of this random search.

Analyse the best (few) parameter setting(s) further.


\section{Results}

All The Graphs.

\section{Conclusion}

Wow island models seem to work so well when you have a large amount of islands.

As it turns out, the size of the islands doesn't matter at all.

Increasing migration rate only seems to be a good idea when increasing offspring ratio at the same time.

Much. Wow.

Future work.


%%%%%%%%%%%%%%%%%%%%%%%%%%%%%%%%%%%%%%%%%%%%%%%%%%%%%%%%%%%%%%%%%%%%%%%%%%%%%%%%





\addtolength{\textheight}{-12cm}   % This command serves to balance the column lengths
                                  % on the last page of the document manually. It shortens
                                  % the textheight of the last page by a suitable amount.
                                  % This command does not take effect until the next page
                                  % so it should come on the page before the last. Make
                                  % sure that you do not shorten the textheight too much.

%%%%%%%%%%%%%%%%%%%%%%%%%%%%%%%%%%%%%%%%%%%%%%%%%%%%%%%%%%%%%%%%%%%%%%%%%%%%%%%%



%%%%%%%%%%%%%%%%%%%%%%%%%%%%%%%%%%%%%%%%%%%%%%%%%%%%%%%%%%%%%%%%%%%%%%%%%%%%%%%%



%%%%%%%%%%%%%%%%%%%%%%%%%%%%%%%%%%%%%%%%%%%%%%%%%%%%%%%%%%%%%%%%%%%%%%%%%%%%%%%%

\begin{thebibliography}{99}

\bibitem{c1} G. O. Young, ÒSynthetic structure of industrial plastics (Book style with paper title and editor),Ó 	in Plastics, 2nd ed. vol. 3, J. Peters, Ed.  New York: McGraw-Hill, 1964, pp. 15Ð64.
\bibitem{c2} W.-K. Chen, Linear Networks and Systems (Book style).	Belmont, CA: Wadsworth, 1993, pp. 123Ð135.
\end{thebibliography}




\end{document}

